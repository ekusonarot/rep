\documentclass[11pt]{jarticle}

\usepackage[dvipdfmx]{graphicx}
\usepackage{listings}

\lstset{
    basicstyle={\ttfamily\small}, %書体の指定
    %frame=tRBl, %フレームの指定
    %framesep=10pt, %フレームと中身(コード)の間隔
    breaklines=true, %行が長くなった場合の改行
    linewidth=16cm, %フレームの横幅
    lineskip=-0.5ex, %行間の調整
    tabsize=2 %Tabを何文字幅にするかの指定
}
\renewcommand{\lstlistingname}{リスト}
\setlength{\oddsidemargin}{-6.35mm}
\setlength{\textwidth}{160.0mm}

\begin{document}

\title{システムプログラミング2\\期末レポート}
\author{09430565\\大橋虎ノ介}
\date{出題日 2020年 10月 5日\\
提出日\number\year 年\number\month 月\number\day 日\\
締切日 2021年 1月 25日}

\maketitle
\begin{center}
教科書\\
 パターソン\&ヘネシー「コンピュータの構成と設計」第5版(上)(下)
\end{center}
\newpage

\section{概要} \label{sec:abstract}

本実験では,アセンブラ言語でプログラミングを行い,C言語におけるヒープ,
スタックとコンピュータアーキテクチャとの関係, main関数以前の動作などについて,実例を通して理解を深める.
本報告書では,設定された以下5つの課題を解き,最終的にprintf関数のサブセット(以降myprintfと呼ぶ)を実装した結果とそれに対する考察を述べる.
実装する出力フォーマット指定子を表\ref{table:ident}に示す.

\begin{itemize}
  \item 課題2-1 教科書A.6節 「手続き呼出し規約」に従って,各種手続きをアセンブラで記述せよ.また,記述した syscalls.s の関数をC言語から呼び出すことで, \ref{sec:program}節リスト\ref{hanoi}ハノイの塔を完成させよ.
  \item 課題2-2 spim-gcc の引数保存に関するスタックの利用方法について,説明せよ. そのことは,規約上許されるスタックフレームの最小値24とどう関係しているか. このスタックフレームの最小値規約を守らないとどのような問題が生じるかについて解説せよ.
  \item 課題2-3 \ref{sec:program}節リスト\ref{report2-1}をコンパイルした結果をもとに,auto変数とstatic変数の違い,ポインタと配列の違いについてレポートせよ.
  \item 課題2-4 printf など,一部の関数は,任意の数の引数を取ることができる. これらの関数を可変引数関数と呼ぶ. MIPSのCコンパイラにおいて可変引数関数の実現方法について考察し,解説せよ.
  \item 課題2-5 printf のサブセットを実装し, SPIM上でその動作を確認する応用プログラム(自由なデモプログラム)を作成せよ.
\end{itemize}

\begin{table}[h]
  \begin{center}
  \label{table:ident}
  \caption{識別子一覧}
    \begin{tabular}{l|l|l}
      指定し & 対応する型 & 説明 \\ \hline
      \%d & int & 整数を10進数で出力する \\
      \%s & char * & 文字列を出力する \\
      \%c & char & 1文字を出力する \\
      \%p & アドレス & アドレスを受け取り16進数で出力する \\
      \%\% & なし & \%を出力する
    \end{tabular}
\end{center}
\end{table}

\section{プログラムの設計方針} \label{sec:policy}

myprintf関数は3つの部分から構成することにした.それぞれについて作成方針を立てる.

\begin{itemize}
  \item myprintf本体(\ref{sec:myprintf}節)
  \item 10進数を16進数に変換するprint\_nbase関数(\ref{sec:16base}節)
  \item 出力処理を行う関数(\ref{sec:io}節)
\end{itemize}

\subsection{myprintf本体} \label{sec:myprintf}

第1引数にフォーマット文字列のアドレス,第2引数以降は出力する変数を受け取る.
文字列を出力する方法としては,フォーマット文字列の一文字目から順番に,1文字ずつ
出力していく.

文字を出力していく中で,"\%"があれば,それに続く識別子に応じて第2引数以降の
変数の値を出力する.

\subsection{10進数から16進数へ変換する関数} \label{sec:16base}

この関数はmyprintf内で呼び出される.第1引数にint型のアドレスのデータを受け取り,
第2引数に変換先の奇数を受け取る.
戻り値はなく,変換した文字列を出力する.

今回の実験では16進数への変換のみを想定している.

\subsection{入出力処理} \label{sec:io}

myprintf関数内で呼び出す入出力関数はアセンブラ言語で記述する.
実装する関数は以下の5つである.

printから始まる関数は整数を出力する関数,1文字を出力する関数そして文字列を出力する関数である.
readから始まる関数は動作確認用のプログラムで使用するユーザの入力を受け付ける関数である.

\begin{itemize}
  \item print\_int
  \item print\_char
  \item print\_string
  \item read\_string
  \item read\_int
\end{itemize}

\section{プログラムリストおよび,その説明} \label{sec:program_exp}

\ref{sec:policy}節の設計方針に従い,myprintf関数を実装した.
myprintf関数およびprint\_nbase関数をリスト\ref{list:myprintf},myprintf関数内で呼び出す
出力関数を\ref{sec:program}節のリスト\ref{syscalls}に示す.

\begin{lstlisting}[label=list:myprintf,caption=myprintf関数,numbers=left]
  void print_nbase(int addr, int base) {
    char i2a[]={'0','1','2','3','4','5','6','7','8',
                '9','A','B','C','D','E','F'};
    int stack[32];
    int idx=0;
    int tmp_base=base;
    print_string("0x");
    for (;addr>=1;idx++) {
        stack[idx]=addr%tmp_base;
        addr/=base;
    }
    idx--;
    for (;idx>=0;idx--) {
        print_char(i2a[stack[idx]]);
    }
  }

  void myprintf(char* fmt, ...) {
    int i;
    void* arg_p = ((char*)&fmt)+((sizeof(fmt)+3) /4) *4;
    for (i=0; fmt[i]!='\0'; i++) {
        if (fmt[i] == '%' && fmt[i+1] != '\0') {
            i++;
            switch (fmt[i]) {
            case 'd':
                print_int(*(int*)arg_p);
                arg_p += ((sizeof(int)+3) /4) *4;
                break;
            case 's':
                print_string((char*)*(char**)arg_p);
                arg_p += ((sizeof(char*)+3) /4) *4;
                break;
            case 'c':
                print_char(*(char*)arg_p);
                arg_p += ((sizeof(char)+3) /4) *4;
                break;
            case 'p':
                print_nbase(*(int*)arg_p,16);
                arg_p += ((sizeof(char*)+3) /4) *4;
                break;
            case '%':
                print_char('%');
                arg_p += ((sizeof(char)+3) /4) *4;
                break;
            default:
                break;
            }
        }else{
            print_char(fmt[i]);
        }
    }
  }
\end{lstlisting}

\section{プログラムの使用方法}

プログラムの動作確認は,spim-gccでコンパイルしたアセンブラコードと
syscalls.s(\ref{sec:program}節,リスト\ref{syscalls})をMIPSシミュレータであるxspim上にロードして動作確認を行う.

xspimコマンドを実行する\\
$>$ xspim -mapped\_io\&\\

loadメニューからファイル名で指定し,runメニューを選んでokを押すことで実行できる.

\section{プログラム作成過程に関する考察}

\ref{sec:policy}節で述べた実装方針に基づいて,\ref{sec:program_exp}節ではその実装を行った.
本節では,myprintfの作成過程において検討した内容,および,考察した内容について述べる.

\subsection{識別子以外の部分の出力について}

現在では,フォーマット文字列の先頭から1文字ずつ出力していく方式にしているが,
初期段階では,"\%"が現れる直前の文字列をまとめて出力するという方法を検討していた.

文字列をまとめて出力する方法を検討していた理由は,1文字ずつ出力するよりも早いと考えたからだ.
しかし,最終的には1文字ずつ出力する方法をとった.
その理由は,文字列として出力する場合,識別子が現れる直前の文字列を保存するバッファを確保するか,
もしくは識別子の"\%"の部分を"$\backslash$0"としなければならない.

バグが混入するのを避けるために複雑な処理をしない現在の方法を採用した.

\section{設問に対する回答}

本節では概要で設定した課題に対する解答を述べる.

\subsection{課題2-1 各種手続きをアセンブラで記述せよ}

文字列や整数の入出力処理をアセンブラで記述し,syscalls.sとした
(\ref{sec:program}節リスト\ref{syscalls}).

hanoi.cをアセンブラにコンパイルしたhanoi.s(\ref{sec:program}節
リスト\ref{hanoi.s})と作成したsyscalls.sをSPIM上にロードし実行した結果
をリスト\ref{hanoi_exec}に示す.

\begin{lstlisting}[caption=実行結果,label=hanoi_exec]
Enter number of disks> 3
Move disk 1 from peg 1 to peg 2.
Move disk 2 from peg 1 to peg 3.
Move disk 1 from peg 2 to peg 3.
Move disk 3 from peg 1 to peg 2.
Move disk 1 from peg 3 to peg 1.
Move disk 2 from peg 3 to peg 2.
Move disk 1 from peg 1 to peg 2.
\end{lstlisting}

\subsection{課題2-2 spim-gcc の引数保存に関するスタックの利用方法について,説明せよ}

spim-gccでは,関数を呼び出す際に第1引数~第4引数を保存するためのスタック領域を余分に確保して関数を呼び出す.
この領域は被呼び出し関数が\$a0~\$a3を保存できるようになっている.

スタックフレームの最小値規約24バイトとは\$fp,\$raそして引数4つ,それぞれ4バイトであるから24バイトを保存するための最小値である.

もしこの規約を守らない関数を呼び出した時は,\$a0~\$a3の保存を自分が確保した領域しか使わないため問題にならない.
しかし,規約を守らない関数が規約を守る関数を呼び出した際は,被呼び出し関数によって自分のために確保した領域を破壊してしまう.

呼び出し側で被呼び出し関数のスタック領域を確保する方法には2つのメリットがある.

1つは被呼び出し関数が\$a0~\$a3をスタックへ保存するか決められる点である.
\$a0~\$a3の書き換えが起こらなければメモリへの書き込みを避け高速化できる.

2つ目は,被呼び出し関数から見て,引数が連続した領域に並ぶ点である.そのため,コンパイラの実装が容易になる.

\subsection{課題2-3 auto変数とstatic変数の違い,ポインタと配列の違いについて}

リスト\ref{report2-1}をコンパイルした結果をリスト\ref{report2-1c}に示す.

リスト\ref{report2-1}の1行目:primes\_statはstatic変数であり,16行目:primes\_autoはauto変数である.
リスト\ref{report2-1c}においてprimes\_statと対応する部分を探すと91行目:.comm \_primes\_stat ,40という行が見つかる.
これはデータセグメントに40バイト確保している.
一方,primes\_autoをリスト\ref{report2-1c}から探すと66行目:.ascii "primes\_auto[0]$\backslash$000"という文字列が見つかる.
\$LC4は82行目に使われており,84行目の\_print\_varの第2引数がprime\_auto[0]の値であると分かる.
83行目の\$a1つまり16(\$fp)はスタック領域であるから,auto変数はスタック領域に保存されていることが分かる.

データセグメントにあるstatic変数はプログラムの開始から終了まで値を保持し続けるが,スタック領域にあるauto変数は
関数が終了するとその領域を解放する.

リスト\ref{report2-1}の3行目:string\_ptrはポインタであり,4行目:string\_aryは配列である.
リスト\ref{report2-1c}においてstring\_ptrと対応する部分は23行目にある.
ここをみると,string\_ptrは18行目"ABCDEFG$\backslash$000"というデータを持った\$LC0へのアドレスを持っていることが分かる.
一方,リスト\ref{report2-1c}において,string\_aryに対応する部分は26行目にある.
ここでは配列が"ABCDEFG$\backslash$000"というでーたを持っていることが分かる.

\subsection{課題2-4 MIPSのCコンパイラにおいて可変引数関数の実現方法について}

C言語では,可変長の引数を扱うために, … を使った構文が用意されている.
spim-gccでは,関数の引数が呼び出した関数のスタック領域に順番に並べられる.

実装するmyprintf関数の第1引数は必須であるので,そのアドレスを元に第2引数,
第3引数を求めることができる.
spim-gccのスタック利用規約から,第2引数は第1引数の4バイト先にあり,
第3引数はさらに4バイト先にあるので,引数のアドレスは以下の式により求められる.
\\
第n引数のアドレス=第n-1引数のアドレス+((sizeof(第n-1引数)+3) /4) *4;
\\
((sizeof(第n-1引数)+3) /4) *4となるのは,spim-gccでのsizeofは3以下の時4の倍数に切り上げるからである.

引数の数と型については,第一引数の\%とsやdなどの識別子をみれば良い.

\subsection{課題2-5 printf のサブセットを実装し, SPIM上でその動作を確認する応用プログラム(自由なデモプログラム)を作成せよ.}

動作確認用にmyprintfを用いてFizzBuzzを書いた(リスト\ref{FizzBuzz}).
FizzBuzzとは,1から任意の数までの数字を順番に表示する.
ただし,3の倍数の時は「Fizz」,5の倍数の時は「Buzz」,3と5の倍数の時は「FizzBuzz」と表示する.

リスト\ref{FizzBuzz}はユーザから整数の入力をもらい,その数字まで
FizzBuzzを表示するプログラムである.

15を入力したときの出力結果をリスト\ref{output}に示す.

\begin{lstlisting}[caption=FizzBuzz,label=FizzBuzz,numbers=left]  
int main() {
    int i, j;
    char Fizz[5]="Fizz", Buzz[5]="Buzz";

    i = read_int();
    for (j=1; j<=i; j++) {
        if (j%3==0 || j%5==0) {
            if (j%3==0) {
                myprintf("%s", Fizz);
            }
            if (j%5==0) {
                myprintf("%s", Buzz);
            }
            myprintf("%c", ' ');
        } else {
            myprintf("%d",j);
            myprintf("%c", ' ');
        }
    }
    myprintf("\n%%%%\n");
    myprintf("Fizz:%p\n",Fizz);
    myprintf("Buzz:%p\n",Buzz);
}
\end{lstlisting}

\begin{lstlisting}[caption=出力結果,label=output]
1 2 Fizz 4 Buzz Fizz 7 8 Fizz Buzz 11 Fizz 13 14 FizzBuzz
%%
Fizz:0x7FFFFD90
Buzz:0x7FFFFD98
  
\end{lstlisting}

\section{プログラムリスト} \label{sec:program}

\begin{lstlisting}[caption=hanoi.c,label=hanoi,numbers=left]
  void hanoi(int n, int start, int finish, int extra)
  {
    if (n != 0){
      hanoi(n - 1, start, extra, finish);
      print_string("Move disk ");
      print_int(n);
      print_string(" from peg ");
      print_int(start);
      print_string(" to peg ");
      print_int(finish);
      print_string(".\n");
      hanoi(n - 1, extra, finish, start);
    }
  }
  
  main()
  {
    int n;
    print_string("Enter number of disks> ");
    n = read_int();
    hanoi(n, 1, 2, 3);
  }
\end{lstlisting}

\begin{lstlisting}[caption=hanoi.s,label=hanoi.s,numbers=left]
	.file	1 "hanoi.c"

 # -G value = 0, Arch = r2000, ISA = 1
 # GNU C version 2.96 20000731 (Red Hat Linux 7.3 2.96-113.2) (mipsel-linux) compiled by GNU C version 2.96 20000731 (Red Hat Linux 7.3 2.96-113.2).
 # options passed:  -mno-abicalls -mrnames -mmips-as
 # -mno-check-zero-division -march=r2000 -O0 -fleading-underscore
 # -finhibit-size-directive -fverbose-asm
 # options enabled:  -fpeephole -ffunction-cse -fkeep-static-consts
 # -fpcc-struct-return -fsched-interblock -fsched-spec -fbranch-count-reg
 # -fnew-exceptions -fcommon -finhibit-size-directive -fverbose-asm
 # -fgnu-linker -fargument-alias -fleading-underscore -fident -fmath-errno
 # -mrnames -mno-check-zero-division -march=r2000


	.rdata
	.align	2
$LC0:
	.asciiz	"Move disk "
	.align	2
$LC1:
	.asciiz	" from peg "
	.align	2
$LC2:
	.asciiz	" to peg "
	.align	2
$LC3:
	.asciiz	".\n"
	.text
	.align	2
_hanoi:
	subu	$sp,$sp,24
	sw	$ra,20($sp)
	sw	$fp,16($sp)
	move	$fp,$sp
	sw	$a0,24($fp)
	sw	$a1,28($fp)
	sw	$a2,32($fp)
	sw	$a3,36($fp)
	lw	$v0,24($fp)
	beq	$v0,$zero,$L3
	lw	$v0,24($fp)
	addu	$v0,$v0,-1
	move	$a0,$v0
	lw	$a1,28($fp)
	lw	$a2,36($fp)
	lw	$a3,32($fp)
	jal	_hanoi
	la	$a0,$LC0
	jal	_print_string
	lw	$a0,24($fp)
	jal	_print_int
	la	$a0,$LC1
	jal	_print_string
	lw	$a0,28($fp)
	jal	_print_int
	la	$a0,$LC2
	jal	_print_string
	lw	$a0,32($fp)
	jal	_print_int
	la	$a0,$LC3
	jal	_print_string
	lw	$v0,24($fp)
	addu	$v0,$v0,-1
	move	$a0,$v0
	lw	$a1,36($fp)
	lw	$a2,32($fp)
	lw	$a3,28($fp)
	jal	_hanoi
$L3:
	move	$sp,$fp
	lw	$ra,20($sp)
	lw	$fp,16($sp)
	addu	$sp,$sp,24
	j	$ra
	.rdata
	.align	2
$LC4:
	.asciiz	"Enter number of disks> "
	.text
	.align	2
main:
	subu	$sp,$sp,32
	sw	$ra,28($sp)
	sw	$fp,24($sp)
	move	$fp,$sp
	la	$a0,$LC4
	jal	_print_string
	jal	_read_int
	sw	$v0,16($fp)
	lw	$a0,16($fp)
	li	$a1,1			# 0x1
	li	$a2,2			# 0x2
	li	$a3,3			# 0x3
	jal	_hanoi
	move	$sp,$fp
	lw	$ra,28($sp)
	lw	$fp,24($sp)
	addu	$sp,$sp,32
	j	$ra
\end{lstlisting}

\begin{lstlisting}[caption=report2-1.c,label=report2-1,numbers=left]
int primes_stat[10];

char * string_ptr   = "ABCDEFG";
char   string_ary[] = "ABCDEFG";

void print_var(char *name, int val)
{
  print_string(name);
  print_string(" = ");
  print_int(val);
  print_string("\n");
}

main()
{
  int primes_auto[10];

  primes_stat[0] = 2;
  primes_auto[0] = 3;

  print_var("primes_stat[0]", primes_stat[0]);
  print_var("primes_auto[0]", primes_auto[0]);
}
\end{lstlisting}

\begin{lstlisting}[caption=report2-1.s,label=report2-1c,numbers=left]
	.file	1 "step4-1.c"

 # -G value = 0, Cpu = r2000, ISA = 1
 # GNU C version 2.96-mips3264-000710 (mipsel-linux) compiled by GNU C version 2.96 20000731 (Red Hat Linux 7.2 2.96-112.7.2).
 # [AL 1.1, MM 40] BSD Mips
 # options passed:  -mno-abicalls -mrnames -mmips-as -mcpu=r2000 -O0
 # -fleading-underscore -finhibit-size-directive -fverbose-asm
 # options enabled:  -fpeephole -fkeep-static-consts -fpcc-struct-return
 # -fsched-interblock -fsched-spec -fnew-exceptions -fcommon
 # -finhibit-size-directive -fverbose-asm -fgnu-linker -flive-range-gdb
 # -fargument-alias -fleading-underscore -fdelay-postincrement -fident
 # -fmath-errno -msplit-addresses -mrnames -mdebugf -mdebugi -mcpu=r2000


	.rdata
	.align	0
	.align	2
$LC0:
	.ascii	"ABCDEFG\000"
	.data
	.align	0
	.align	2
_string_ptr:
	.word	$LC0
	.align	2
_string_ary:
	.ascii	"ABCDEFG\000"
	.rdata
	.align	0
	.align	2
$LC1:
	.ascii	" = \000"
	.align	2
$LC2:
	.ascii	"\n\000"
	.text
	.align	2
	.set	nomips16
_print_var:
	subu	$sp,$sp,24
	sw	$ra,20($sp)
	sw	$fp,16($sp)
	move	$fp,$sp
	sw	$a0,24($fp)
	sw	$a1,28($fp)
	lw	$a0,24($fp)
	jal	_print_string
	la	$a0,$LC1
	jal	_print_string
	lw	$a0,28($fp)
	jal	_print_int
	la	$a0,$LC2
	jal	_print_string
	move	$sp,$fp
	lw	$ra,20($sp)
	lw	$fp,16($sp)
	addu	$sp,$sp,24
	j	$ra
	.rdata
	.align	0
	.align	2
$LC3:
	.ascii	"primes_stat[0]\000"
	.align	2
$LC4:
	.ascii	"primes_auto[0]\000"
	.text
	.align	2
	.set	nomips16
main:
	subu	$sp,$sp,64
	sw	$ra,60($sp)
	sw	$fp,56($sp)
	move	$fp,$sp
	li	$v0,2			# 0x2
	sw	$v0,_primes_stat
	li	$v0,3			# 0x3
	sw	$v0,16($fp)
	la	$a0,$LC3
	lw	$a1,_primes_stat
	jal	_print_var
	la	$a0,$LC4
	lw	$a1,16($fp)
	jal	_print_var
	move	$sp,$fp
	lw	$ra,60($sp)
	lw	$fp,56($sp)
	addu	$sp,$sp,64
	j	$ra

	.comm	_primes_stat,40
\end{lstlisting}

\begin{lstlisting}[caption=syacalls.s,label=syscalls,numbers=left]
  .text
  .align  2
_print_int:
  subu    $sp, $sp, 24
  sw      $ra, 20($sp)
  sw      $fp, 16($sp)

  li      $v0, 1
  syscall

  lw      $fp, 16($sp)
  lw      $ra, 20($sp)
  addi    $sp, $sp, 24

  j       $ra

  .align  2
_print_string:
  subu    $sp, $sp, 24
  sw      $ra, 20($sp)
  sw      $fp, 16($sp)

  li      $v0, 4
  syscall

  lw      $fp, 16($sp)
  lw      $ra, 20($sp)
  addi    $sp, $sp, 24

  j       $ra


  .align  2
_read_int:
  subu    $sp, $sp, 24
  sw      $ra, 20($sp)
  sw      $fp, 16($sp)

  li      $v0, 5
  syscall

  lw      $fp, 16($sp)
  lw      $ra, 20($sp)
  addi    $sp, $sp, 24

  j       $ra

  

  .align  2
_read_string:
  subu    $sp, $sp, 24
  sw      $ra, 20($sp)
  sw      $fp, 16($sp)

  li      $v0, 8
  syscall

  lw      $fp, 16($sp)
  lw      $ra, 20($sp)
  addi    $sp, $sp, 24

  j       $ra
\end{lstlisting}

\end{document}
